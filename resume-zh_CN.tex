% !TEX TS-program = xelatex
% !TEX encoding = UTF-8 Unicode
% !Mode:: "TeX:UTF-8"

\documentclass{resume}
\usepackage{zh_CN-Adobefonts_external} % Simplified Chinese Support using external fonts (./fonts/zh_CN-Adobe/)
% \usepackage{NotoSansSC_external}
% \usepackage{NotoSerifCJKsc_external}
% \usepackage{zh_CN-Adobefonts_internal} % Simplified Chinese Support using system fonts
\usepackage{linespacing_fix} % disable extra space before next section
\usepackage{cite}

\begin{document}
\pagenumbering{gobble} % suppress displaying page number

\name{王嘉梁}

\basicInfo{
  \email{jw2476@cornell.edu} \textperiodcentered\
  \phone{(+86) 173-1946-6814} \textperiodcentered\
  \linkedin[wangvincent1993]{https://www.linkedin.com/in/wangvincent1993}}
 
\section{\faGraduationCap\  教育背景}
\datedsubsection{\textbf{康奈尔大学}, 美国}{2017 -- 2018}
\textit{硕士研究生}\ 计算机科
\datedsubsection{\textbf{伊利诺伊大学香槟分校}, 美国}{2012 -- 2016}
\textit{学士}\ 计算机科学,物理(双学位),数学(辅修)

\section{\faUsers\ 实习/项目经历}
\datedsubsection{\textbf{商汤绝影} 上海}{2024年5月 -- 至今}
\role{专家算法工程师}{工作经历}
绝影智驾云端算法开发
\begin{itemize}
  \item 负责云端地图场景挖掘系统,利用高精地图数据,Shapely库,道路几何关系,开发60+种道路场景打标算法。
  \item 负责车端地图场景挖掘系统,基于SD地图信息,在车端开发基于道路几何的实时化打标算法。
  \item 参与开发云端自动标注算法,利用高精地图反投影,开发道路拓扑自动标注算法,并实现BEV视角可视化。
\end{itemize}

\datedsubsection{\textbf{集度汽车} 上海}{2022年11月 -- 2024年5月}
\role{专家算法工程师}{工作经历}
集度汽车智驾感知算法开发
\begin{itemize}
  \item 负责集度汽车哨兵功能可视化,使用opencv解析数据并回灌感知融合结果,可快速验证算法有效性。 
  \item 负责集度汽车车端哨兵视觉感知pipeline,利用OpenCV, Eigen等将鱼眼视角转成柱状图视角的方法,消除畸变,适配传统CV算法。
  \item 负责集度智驾云端场景挖掘系统框架,利用大模型挖掘20+静态场景标签,利用规则算法挖掘20+动态场景标签。
\end{itemize}

\datedsubsection{\textbf{小马智行} 上海}{2019年9月 -- 2022年9月}
\role{高级软件开发工程师}{工作经历}
小马智行数据开发
\begin{itemize}
  \item 负责标注pipeline的数据导出模块,利用crobjob完成每日定时任务,交付感知团队标注数据。
  \item 参与标注pipeline的数据生成模块,利用python, C++完成数据解析,数据抽帧,并利用API回传结果。
  \item 参与开发标注任务管理模块后端API,利用Flask完成后端任务创建与任务查询功能。
\end{itemize}

\datedsubsection{\textbf{滴滴出行} 北京}{2018年7月 -- 2019年9月}
\role{软件开发工程师}{工作经历}
滴滴出行国际化后端开发
\begin{itemize}
  \item 通过分析进线的主要问题,协助产品经理推出产品方案,并完成技术方案的开发与落地,减少单项乘客侧订单投诉率(CPO)
\end{itemize}


% Reference Test
%\datedsubsection{\textbf{Paper Title\cite{zaharia2012resilient}}}{May. 2015}
%An xxx optimized for xxx\cite{verma2015large}
%\begin{itemize}
%  \item main contribution
%\end{itemize}

\section{\faCogs\ IT 技能}
% increase linespacing [parsep=0.5ex]
\begin{itemize}[parsep=0.5ex]
  \item 技术栈: C/C++, Python, Protobuf, Bazel, Kubernetes, MySql
  \item 证书: PMP
\end{itemize}

%% Reference
%\newpage
%\bibliographystyle{IEEETran}
%\bibliography{mycite}
\end{document}
