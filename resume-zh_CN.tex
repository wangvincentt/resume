% !TEX TS-program = xelatex
% !TEX encoding = UTF-8 Unicode
% !Mode:: "TeX:UTF-8"

\documentclass{resume}
\usepackage{zh_CN-Adobefonts_external} % Simplified Chinese Support using external fonts (./fonts/zh_CN-Adobe/)
% \usepackage{NotoSansSC_external}
% \usepackage{NotoSerifCJKsc_external}
% \usepackage{zh_CN-Adobefonts_internal} % Simplified Chinese Support using system fonts
\usepackage{linespacing_fix} % disable extra space before next section
\usepackage{cite}

\begin{document}
\pagenumbering{gobble} % suppress displaying page number

\name{王嘉梁}

\basicInfo{
  \email{jw2476@cornell.edu} \textperiodcentered\
  \phone{(+86) 173-1946-6814} \textperiodcentered\
  \linkedin[wangvincent1993]{https://www.linkedin.com/in/wangvincent1993}}
 
\section{\faGraduationCap\  教育背景}
\datedsubsection{\textbf{康奈尔大学}, 美国}{2017 -- 2018}
\textit{硕士研究生}\ 计算机科
\datedsubsection{\textbf{伊利诺伊大学香槟分校}, 美国}{2012 -- 2016}
\textit{学士}\ 计算机科学,物理(双学位),数学(辅修)

\section{\faUsers\ 实习/项目经历}
\datedsubsection{\textbf{商汤绝影} 上海}{2024年5月 -- 至今}
\role{专家算法工程师}{工作经历}
绝影智驾云端算法开发
\begin{itemize}
  \item 基于高精地图反投影,开发道路拓扑自动标注算法,并实现BEV视角可视化。
  \item 基于高精地图,开发60+种道路场景的云端自动化打标算法。
  \item 基于SD地图,开发车端实时自动化打标算法。
\end{itemize}

\datedsubsection{\textbf{集度汽车} 上海}{2022年11月 -- 2024年5月}
\role{专家算法工程师}{工作经历}
集度汽车智驾感知算法开发
\begin{itemize}
  \item 开发集度汽车车端哨兵视觉感知pipeline,包括视觉预处理与后处理算法。
  \item 开发集度汽车云端哨兵可视化,可回放感知融合结果。
  \item 开发集度智驾云端场景挖掘系统框架并提供规则与模型挖掘算法。
\end{itemize}

\datedsubsection{\textbf{小马智行} 上海}{2019年9月 -- 2022年9月}
\role{高级软件开发工程师}{工作经历}
小马智行数据开发
\begin{itemize}
  \item 开发自动标注数据管道,包括数据生成和数据导出模块,为下游算法团队提供可靠的标注数据。
  \item 开发Console模块后端API,提供提供项目负责人一站化服务,包括提交任务,监控项目进度和查询导出数据。
\end{itemize}

\datedsubsection{\textbf{滴滴出行} 北京}{2018年7月 -- 2019年9月}
\role{软件开发工程师}{工作经历}
滴滴出行国际化后端开发
\begin{itemize}
  \item 通过分析进线的主要问题,影响产品经理推出产品方案;作为技术Owner,协调各模块提供技术方案,并促成产品交付;通过产品技术方案,减少单项乘客侧订单投诉率(CPO)
\end{itemize}


% Reference Test
%\datedsubsection{\textbf{Paper Title\cite{zaharia2012resilient}}}{May. 2015}
%An xxx optimized for xxx\cite{verma2015large}
%\begin{itemize}
%  \item main contribution
%\end{itemize}

\section{\faCogs\ IT 技能}
% increase linespacing [parsep=0.5ex]
\begin{itemize}[parsep=0.5ex]
  \item 技术栈: C/C++, Python, Protobuf, Bazel, Kubernetes
  \item 证书: PMP
\end{itemize}

%% Reference
%\newpage
%\bibliographystyle{IEEETran}
%\bibliography{mycite}
\end{document}
