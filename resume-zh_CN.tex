% !TEX TS-program = xelatex
% !TEX encoding = UTF-8 Unicode
% !Mode:: "TeX:UTF-8"

\documentclass{resume}
\usepackage{zh_CN-Adobefonts_external} % Simplified Chinese Support using external fonts (./fonts/zh_CN-Adobe/)
% \usepackage{NotoSansSC_external}
% \usepackage{NotoSerifCJKsc_external}
% \usepackage{zh_CN-Adobefonts_internal} % Simplified Chinese Support using system fonts
\usepackage{linespacing_fix} % disable extra space before next section
\usepackage{cite}

\begin{document}
\pagenumbering{gobble} % suppress displaying page number

\name{王嘉梁}

\basicInfo{
  \email{jw2476@cornell.edu} \textperiodcentered\
  \phone{(+86) 173-1946-6814} \textperiodcentered\
  \linkedin[wangvincent1993]{https://www.linkedin.com/in/wangvincent1993}}
 
\section{\faGraduationCap\  教育背景}
\datedsubsection{\textbf{康奈尔大学}, 美国}{2017 -- 2018}
\textit{硕士研究生}\ 计算机科
\datedsubsection{\textbf{伊利诺伊大学香槟分校}, 美国}{2012 -- 2016}
\textit{学士}\ 计算机科学,物理(双学位),数学(辅修)

\section{\faUsers\ 实习/项目经历}
\datedsubsection{\textbf{小马智行} 上海}{2019年9月 -- 至今}
\role{软件开发工程师}{工作经历}
滴滴出行国际化后端开发
\begin{itemize}
  Build Task Generation data-pipeline for Pony.ai labeling platform to process raw data to support labeling team
• Build Exporter data-pipeline to export labeled data to support perception team
• Build Console service to manage user, project, request and task data.
• Develope Console service API to manage data access for other parties for data security
• Reduce resources space by 45% by designing system with part sync and retention policy
• Reduce upload time by 20% by building zip and unzip with retention on Aliyun OSS
  \item 开发Pony.ai的数据管道,去生成任务,标注任务,导出任务,为下游算法团队提供可靠的标注数据。
  \item 开发Console模块,提供项目负责人一站化服务, 包括提交任务,监控项目进度,掌控项目存储目标。
  \item 开发离线任务计算,使下游不再接触线上数据库,保证了数据库的安全以及下游数据的需求。
  \item 开发支持中美数据隔离,保证下游数据最大可用的情况下减少不必要的数据冗余,减少了总数据量的33\%。
\end{itemize}

\datedsubsection{\textbf{滴滴出行} 北京}{2018年7月 -- 2019年9月}
\role{软件开发工程师}{工作经历}
滴滴出行国际化后端开发
\begin{itemize}
  \item 通过分析进线的主要问题,影响产品经理推出产品方案;作为技术Owner,协调各模块提供技术方案,并促成产品交付;通过产品技术方案,减少单项乘客侧订单投诉率(CPO)
  \item 作为"快速开国开城”项目的一部分,从零设计场景与功能管理模块,实现模块代码开发,部署上线,A/B测试以及放量发布;有效提升中台微服务开发效率,提高中台接入满意度
  \item 建设场景与功能模块研发测试规范,搭建资源与流量监控,上线降级开关,总结问题定位经验并输出模块问题快速定位手册,使模块稳定性达到99.99\%
  \item 建设场景接入标准与推广,顺利使两个微服务通过这个标准接入中台
  \item 创建部门分享会,并设立分享会制度,鼓励同事间想法与知识分享
\end{itemize}

\datedsubsection{\textbf{蚂蚁金服(美国)}}{2015 年6月 -- 2015 年8月}
\role{软件开发实习生}{实习项目}
\begin{onehalfspacing}
图像识别与深度学习算法
\begin{itemize}
  \item 通过深度学习技术来识别I型糖尿病的程度
  \item 进行图像预处理来突出眼部血管,并利用交叉验证解决非均衡数据以及过拟合问题
  \item 利用7万张图片,利用Caffe技术做出深度神经网络模型,利用C5.0建立决策树,并将深度神经网络模型与决策树结合,进行预测,并获得Top25\%
\end{itemize}
\end{onehalfspacing}

% Reference Test
%\datedsubsection{\textbf{Paper Title\cite{zaharia2012resilient}}}{May. 2015}
%An xxx optimized for xxx\cite{verma2015large}
%\begin{itemize}
%  \item main contribution
%\end{itemize}

\section{\faCogs\ IT 技能}
% increase linespacing [parsep=0.5ex]
\begin{itemize}[parsep=0.5ex]
  \item 编程语言: C/C++, Python, Golang
  \item 开发工具: Git, Intellij, kubernetes
\end{itemize}

\section{\faInfo\ 其他}
% increase linespacing [parsep=0.5ex]
\begin{itemize}[parsep=0.5ex]
  \item 证书: FRM Part II(通过), CFA I(通过), CQF
  \item GitHub: https://github.com/wangvincentt
\end{itemize}

%% Reference
%\newpage
%\bibliographystyle{IEEETran}
%\bibliography{mycite}
\end{document}
